%%%%%%%%%%%%%%%%%%%%%%%%%%%%%%%%%%%%%%%%%%%%%%%%%%%%%%%%%%%%%%%%%%%%%%%%
% Plantilla TFG/TFM
% Escuela Politécnica Superior de la Universidad de Alicante
% Realizado por: Jose Manuel Requena Plens
% Contacto: info@jmrplens.com / Telegram:@jmrplens
%%%%%%%%%%%%%%%%%%%%%%%%%%%%%%%%%%%%%%%%%%%%%%%%%%%%%%%%%%%%%%%%%%%%%%%%

%\chapter{Introducción (Con ejemplos de contenido)}
\chapter{Introducción}
%Antes de comenzar la lectura de este documento debo agradecer el trabajo realizado por Pedro Pernías Peco en su plantilla de ``tfg'' que se puede ver en \url{https://github.com/lcg51/tfg}. Gracias a esa plantilla me he lanzado a crear mi versión. Algunos contenidos aquí mostrados han sido extraídos de la plantilla de Pedro. 
%\\
%\par Esta plantilla se ha diseñado de 0 y por ello no utiliza la misma estructura que la plantilla de Pedro. Pero la estructura de contenido para un TFG/TFM es la misma y a continuación se muestran las diferentes partes que debe tener un TFG/TFM redactado por Pedro.
Las aplicaciones de dispositivos móviles se han convertido en una parte importante y esencial en nuestras vidas cotidianas, alcanzando casi 96\% de la población en paises desarrollados entre la poblacion joven \citep{atske-2025}. Este fenómeno ha provocado un auge tanto de las tecnologías nativas del móvil, como las de multiplataforma, alcanzando así una mayor cuota del mercado.
\par Este trabajo tiene como objetivo estudiar el framework multiplataforma Flutter y el proceso de desarrollo de una aplicación compatible con los navegadores Web y el sistema operativo Android.

\section{El proceso de enseñanza musical}

%Este texto está escrito pensando en orientar a los alumnos que usarán \LaTeX~para escribir su \gls{tfg} y \gls{tfm}. 
%\\
%\par Contiene información útil para aquellos que no tengan experiencia previa en \LaTeX~así como algunos datos acerca de cómo escribir mejor su \gls{tfg}.
%A continuación, se ofrece una copia de la información que hay en el libro de estilo para la realización de los \gls{tfg} de la EPS de la Universidad de Alicante.
%
%En los capítulos siguientes encontrarás ejemplos de muchas de las cosas que se pueden realizar con \LaTeX. Con un poco de paciencia, estudia cómo se hacen estas cosas y luego aplícalas en tus documentos.

La música, en su manifestación, es una creación artística. Sin embargo, hoy en día, la enseñanza musical, aparte del arte, incluye el estudio teórico, la memorización y repetición de patrones. En España, es una asignatura de estudios primarios y secundarios obligatorios, aunque con el contenido impartido limitado a la lectura de partituras y el uso de instrumentos como la flauta o el ukulele.
\par Por otro lado, en España también existen docenas de conservatorios y escuelas musicales que tienen un programa de enseñanza más amplio y regulado por el Gobierno de España y de la Comunidad Autónoma. Dichas instituciones de enseñanza musical ofrecen varios niveles de profesionalización, siendo los estudios superiores de música el nivel más alto.
\par Este \gls{tfg} está destinado a las personas estudiantes de música que poseen conocimientos teóricos elementales y profesionales sobre la lectura y reconocimiento de las partituras y del lenguaje musical, que se alcanza durante la realización de estudios por las personas de edades de 8 a 18 años. Según el \citet{rd-1577-2006}, los alumnos deben tener, entre otras, las siguientes habilidades:
\begin{itemize}
	\item Desarrollar el oído interno tanto en el análisis como en la realización de ejercicios escritos.
	\item Utilizar el «oído interno» como base de la afinación, de la audición armónica y de la interpretación musical.
	\item Identificar a través de la audición:
	\begin{itemize}
		\item Notas
		\item Intervalos
		\item Acordes
		\item Modos
	\end{itemize}
\end{itemize}

\section{Entrenamiento del oído}
El entrenamiento auditivo es una actividad de desarrollo de la destreza que le permite al músico a reconocer distintas notas, intervalos y modos al escucharlos, sin tener acceso a ninguna herramienta ni partitura. El entrenamiento del oído tiene muchas similtudes con el uso del oído en la vida cotidiana: su propósito es diferenciar y entender auditivamente. En el caso de la música, son los sonidos de un instrumento; en la vida cotidiana serían las palabras e intonaciones de la voz.
\par Dentro de este trabajo, se tratará sobre los elementos auditivos puramente sonoros, sin considerar los aspectos artísticos y emocionales que transmite la música. Los elementos esenciales del sonido son: \todo{Cambiar sin negatividad}
\begin{itemize}
	\item La altura: definida con un nombre (Do, Re; A, B)
	\item La intensidad
	\item La duración
\end{itemize}
\par También se van a considerar las estructuras como:
\begin{itemize}
	\item Estructuras melódicas: modos y escalas
	\item Estructuras armónicas: tonalidades, acordes
	\item Ritmos
\end{itemize}
\par A continuación se definen los conceptos esenciales del sonido y música.
\subsection{Notas musicales}
Las notas musicales son sonidos con una determinada \gls{hz}. La altura de una nota es directamente proporcional a su frecuencia. Cada nota tiene su propio nombre:
\begin{figure}[h]
	\centering
	\begin{subfigure}[b]{0.4\textwidth}
		\centering
		\includegraphics[width=4cm]{archivos/notas}
		\caption{Sistema solfeo}
		\label{fig:notas}
	\end{subfigure}
	~ % Añadir el espacio deseado, si se deja la linea en blanco la siguiente subfigura ira en una nueva linea
	\begin{subfigure}[b]{0.4\textwidth}
		\centering
		\includegraphics[width=4cm]{archivos/notas-anglo}
		\caption{Sistema anglo-sajón}
		\label{fig:notas-anglo}
	\end{subfigure}
	\caption{Nombres de las notas musicales}\label{notas}
\end{figure}
\par En este trabajo, se utilizará el sistema solfeo para referirse a las notas. Cada nota musical también puede representarse en un ordenador a través del protocólo \gls{midi}, tema que se abordará en los siguientes capítulos.
\newpage
\subsection{Instrumento utilizado}
Para la mayoría de los estudiantes del conservatorio, el piano se convierte en el primer instrumento con el cual	se acostumbran a lo largo de su formación musical. Incluso si el estudiante decide especializarse en otro instrumento, el piano resulta fundamental para impartir las asignaturas de teoría musical y el solfeo \footnote{En este \gls{tfg}, el solfeo se refiere al entrenamiento del oído interno y a la lectura de las partituras}.
\par Para el desarrollo de una aplicación de entrenamiento del oído, el piano es una opción adecuada, ya que es un instrumento que se puede virtualizar y repsesentar las notas musicales con precisión y exactitud, incluso cuando el usuario no dispone de periféricos.
\subsection{Octavas}
En el piano hay 88 teclas, cada una de las cuales tiene una frecuecia determinada (desde 20 {hz} y hasta 20000 \gls{hz}). La distancia mínima en frecuencia/altura de cada tecla siempre es la misma, y se denomina "semitono". Gracias a los tonos y semitonos, se puede saber exactamente la distancia entre cada nota, por ejemplo: entre Do y Re la distancia es un tono, entre Do y Do\musSharp{} la distancia es un semitono (ver figura \ref{fig:octavas}).

\begin{figure}[h]
	\centering
	\includegraphics[width=13cm]{archivos/octavas}
	\caption{Las octavas del piano \citep{octavas}}
	\label{fig:octavas}
\end{figure}
\newpage
\subsubsection{Intervalos}
Un intervalo es la diferencia entre dos sonidos aislados, y se puede referir tanto a los acórdes (suenan simultáneamente) como a las melodías (sonidos sucesivos).
\par Los dos tipos de semitonos que existen según el intervalo son:
\begin{itemize}
	\item \textbf{Semitono cromático:} Las dos notas del intervalo tienen el mismo nombre (Ver \ref{fig:semitonos}. Intervalo entre La-La\musSharp{}).
	\item \textbf{Semitono diatónico:} Las dos notas del intervalo tienen nombres distintos (Ver \ref{fig:semitonos}. Intervalo entre La-Si\musFlat{}). 
\end{itemize}
\begin{figure}[h]
	\centering
	\includegraphics[width=9cm]{archivos/semitonos}
	\caption{Los intervalos de semitonos \citep{teoria-semitonos}}
	\label{fig:semitonos}
\end{figure}
\par Los intervalos de uno o más semitonos se clasifican en:
% EJEMPLO 1
\begin{table}[ht]
	\centering
	{\scalefont{0.9}
		\begin{tabular}{@{}llc@{}}
			\toprule
			Nombre		& Cantidad de semitonos	\\ \midrule
			2ª Menor	& 1 Semitono        	\\
			2ª Mayor	& 1 Tono      			\\
			3ª Menor	& 1 Tono + 1 Semitono   \\
			3ª Mayor	& 2 Tonos      			\\
			4ª Justa	& 2 Tonos + 1 Semitono	\\
			5ª Justa 	& 3 Tonos + 1 Semitono	\\
			6ª Menor	& 4 Tonos	 			\\
			6ª Mayor	& 4 Tonos + 1 Semitono  \\
			7ª Menor   	& 5 Tonos       		\\
			7ª Mayor  	& 5 Tonos + 1 Semitono  \\
			8ª Justa  	& 6 Tonos   			\\ \bottomrule
		\end{tabular}
	}
	\caption{Los intervalos en música.}
	\label{tablaintervalos}
\end{table}
\subsection{Melodía}
La melodía se define como una secuencia de notas o tonos con sus duraciones rítmicas determinadas. Dentro de una obra musical, la melodía se mantiene reconocible, y es el elemento que identifica la obra como tal. Un ejemplo claro de melodía sería la parte que interpreta el vocalista principal en una aria de Mozart, que se diferencia claramente del acompañamiento.
\par Los principales componentes de una melodía son la altura de la nota y su duración. Además, la melodía se construye sobre un modo específico, lo que le proporciona un carácter tonal y emocional.
\subsubsection{Escalas}
Se define una escala musical como un conjunto de notas consecutivas ascendentes o descendentes clasificadas por su altura, normalmente teniendo el tamaño de una octava.
Las escalas sirven para construir obras musicales al igual que un pintor elige la paleta para su cuadro: mientras que el compositor selecciona un conjunto de notas, en el pintor elige un grupo de colores.
\par Cada denominacion de cada escala se compone de dos elementos:
\begin{itemize}
	\item La tónica: es la nota inicial que da el nombre a la escala. También puede ser un sostenido (\musSharp{}) o un bemol (\musFlat{}).
	\item El modo: define el carácter de la escala u obra.
\end{itemize}
\paragraph{Modos}
\par En el sistema músico occidental, las notas dentro de cada escala se organizan segun el modo de la escala. 
\par En música existen dos calidades emocionales: Mayor y Menor. La calidad mayor proporciona una apariencia "feliz" y "alegre" de la obra, mientras que el modo menor la hace "triste" y "melancólica".


\subsection{Armonía}
\subsubsection{Tonalidades}
\subsubsection{Acordes}
\subsection{Ritmo}
%\section{Estructura de un \glsentryshort{tfg}}
\section{Problemas de enseñanza tradicional}

%En caso de que el \gls{tfg}/\gls{tfm} tenga como finalidad la elaboración de un proyecto o un 
%informe científico o técnico, deberá ajustarse a lo dispuesto en las normas UNE 
%157001:2002 y UNE 50135:1996 respectivamente.
%
%Si el \gls{tfg}/\gls{tfm} tiene por finalidad la elaboración de un trabajo monográfico, el 
%documento presentado deberá constar de las siguientes partes, teniendo como base la 
%norma UNE 50136:1997.

%\begin{description}
%\item[Preámbulo:] se describirán brevemente la motivación que ha originado la realización del \gls{tfg}/\gls{tfm}, así como una breve descripción de los objetivos generales que se quieren alcanzar con el trabajo presentado.
%\item[Agradecimientos:] se podrán añadir las hojas necesarias para realizar los agradecimientos, a veces obligatorios, a las entidades y organismos colaboradores.
%\item[Dedicatoria:] se podrá añadir una única hoja con dedicatorias, su alineación será derecha.
%\item[Citas:] (frases célebres) se podrá añadir una única hoja con citas, su alineación será derecha.
%\item[Índices:] cada índice debe comenzar en una nueva página, se incluirán los índices que se estimen necesarios (conforme UNE 50111:1989), en este orden:
%\begin{description}
%\item[Índice de contenidos:] (obligatorio siempre) se incluirá un índice de las secciones de las que se componga el documento, la numeración de las 
%divisiones y subdivisiones utilizarán cifras arábigas (según UNE 50132:1994) y harán mención a la página del documento donde se ubiquen.
%\item[Índice de figuras:] si el documento incluye figuras se podrá incluir también un índice con su relación, indicando la página donde se ubiquen.
%\item[Índice de tablas:] en caso de existir en el texto, ídem que el anterior.
%\item[Índice de abreviaturas, siglas, símbolos, etc.:] en caso de ser necesarios se podrán incluir cada uno de ellos.
%\end{description}
%\item[Cuerpo del documento:] en el contenido del documento se da flexibilidad para su organización y se puede estructurar en las secciones que se considere. En todo caso obligatoriamente se deberá, al menos, incluir los siguientes contenidos:
%\begin{description}
%\item[Introducción:] donde se hará énfasis a la importancia de la temática, su vigencia y actualidad; se planteará el problema a investigar, así como el propósito o finalidad de la investigación.
%\item[Marco teórico o Estado del arte:] se hará mención a los elementos conceptuales que sirven de base para la investigación, estudios previos relacionados con el problema planteado, etc.
%\item[Objetivos:] se establecerán el objetivo general y los específicos.
%\item[Metodología:] se indicarán el tipo o tipos de investigación, las técnicas y los procedimientos que serán utilizados para llevarla a cabo; se identificarán la población y el tamaño de la muestra así como las técnicas e instrumentos de recolección de datos.
%\item[Resultados:] incluirá los resultados de la investigación o trabajo, así como el análisis y la discusión de los mismos.
%\end{description}
%\item[Conclusiones:] obligatoriamente se incluirá una sección de conclusiones donde se realizará un resumen de los objetivos conseguidos así como de los resultados obtenidos si proceden.
%\item[Bibliografía y referencias:] se incluirá también la relación de obras y materiales consultados y empleados en la elaboración de la memoria del \gls{tfg}/\gls{tfm}. La bibliografía y las referencias serán indexadas en orden alfabético (sistema nombre y fecha) o se numerará correlativamente según aparezca (sistema numérico). Se empleará la familia 1 como tipo de letra. Podrá utilizarse cualquier sistema bibliográfico normalizado predominante en la rama de conocimiento, estableciéndose como prioritarios el sistema ISO 690, sistema \gls{apa}  o Harvard (no necesariamente en ese orden de preferencia). En esta plantilla Latex se propone usar el estilo \gls{apa} indicándolo en la línea correspondiente como 
%\begin{verbatim}
%\bibliographystyle{apacite}
%\end{verbatim}
%
%
%\item[Anexos:] se podrán incluir los anexos que se consideren oportunos.
%
%\end{description}

Actualmente, la forma más eficiente del aprendizaje de la música, y la preferida por los tutores, es la asistencia a los conservatorios y escuelas de música de forma extracurricular. En consecuencia, el niño es obligado a asistir tanto a su escuela de primaria/secundaria como, durante su tiempo libre, a la escuela de música. Además, es obligado a la realización de las tareas academicas de ambas durante su tiempo libre, lo cual podría generar falta de motivación, resentimiento y el síndrome de agotamiento (\textit{burnout}). Por otro lado, los alumnos también informan de la falta de reconocimiento y la falta de apoyo \citep{orzel-2010}.
\par Otro problema relevante está relacionado con los estudiantes de música procedentes de familias de bajos recursos económicos. Según \citet{busby-2019}, los niños procedentes de familias pobres tienen 3 veces más probabilidades de no participar en actividades extracurriculares, como los estudios musicales. En contraste, la mayoría de los padres de familias de bajos ingresos reconocen el efecto positivo que la música y la formación musical tienen sobre sus hijos \citep{ho-2020}.

%\section{Apartados dentro de los capítulos}
%En \LaTeX~existen diferentes niveles de títulos para realizar secciones, subsecciones, etc. En esta web puedes ver más información al respecto \url{https://en.wikibooks.org/wiki/LaTeX/Document_Structure}
%
%Para ello se utilizan los siguientes comandos;
%
%\begin{lstlisting}[style=Latex-color]
%	\section{Esto es una sección}
%	Y este el contenido de la sección.
%	\subsection{Esto es una subsección}
%	Y este el contenido de la subsección.
%	\subsubsection{Esto es una subsubsección}
%	Y este el contenido de la subsubsección.
%	\paragraph{Esto es un paragraph}
% 	Y este el contenido del paragraph. Que siempre se inicia en la misma línea que el título del mismo.
%\end{lstlisting}
% Y se genera lo siguiente:
% \section{Esto es una sección}
%	Y este el contenido de la sección.
%	\subsection{Esto es una subsección}
%	Y este el contenido de la subsección.
%	\subsubsection{Esto es una subsubsección}
%	Y este el contenido de la subsubsección.
%	\paragraph{Esto es un paragraph}
% 	Y este el contenido del paragraph. Que siempre se inicia en la misma línea que el título del mismo.

%\section{Citar bibliografía}
%Para citar la bibliografía tal como se define en el sistema APA (en esta web se indica como debe aparecer en el texto la cita: \url{http://guides.libraries.psu.edu/apaquickguide/intext}) se debe realizar con alguno de los comandos mostrados a continuación:
%
%\begin{lstlisting}[style=Latex-color]
%Esto es una cita estándar: \citet{Shaw1996}, que también puedes mostrar con paréntesis así: \citep{Shaw1996}. También se puede realizar una cita indicando a qué parte te refieres \citep[ver][Cap. 2]{Shaw1996} o \citep[Cap. 2]{Shaw1996} o \citep[ver][]{Shaw1996}. 
%
%También puedes mostrar todos los autores cuando hay más de 2 autores añadiendo un asterisco después del comando como: \citet*{Akyildiz2005}, sin el asterisco quedaría así: \citet{Akyildiz2005}.
%
%O puedes citar dos o más fuentes al mismo tiempo: \citep{Barkan1995,Leighton2012}
%
%\end{lstlisting}
%Y \LaTeX~genera lo siguiente:
%\\
%\par Esto es una cita estándar: \citet{Shaw1996}, que también puedes mostrar con paréntesis así: \citep{Shaw1996}. También se puede realizar una cita indicando a qué parte te refieres \citep[ver][Cap. 2]{Shaw1996} o \citep[Cap. 2]{Shaw1996} o \citep[ver][]{Shaw1996}. 
%\\
%\par También puedes mostrar todos los autores cuando hay más de 2 autores añadiendo un asterisco después del comando como: \citet*{Akyildiz2005}, sin el asterisco quedaría así: \citet{Akyildiz2005}.
%\\
%\par O puedes citar dos o más fuentes al mismo tiempo: \citep{Barkan1995,Leighton2012}
%
%
%\section{Notas a pie de página}
%
%Para introducir notas a pie de página se debe escribir lo siguiente:
%
%\begin{lstlisting}[style=Latex-color]
%	La plantilla necesita el motor XeLaTeX \footnote{Para más información sobre XeLaTeX visita \url{https://es.sharelatex.com/learn/XeLaTeX}} (el más recomendable actualmente), por lo que si el programa que utilizas compila la plantilla con el motor pdfLaTeX \footnote{También puedes buscar más información en internet} (el más habitual pero menos potente) debes cambiarlo por XeLaTeX en las opciones del programa. Si no sabes como hacerlo busca en el manual del programa o en google.
%\end{lstlisting}
%
%\LaTeX~genera lo siguiente (observa las notas a pie de página):
%\\
%\par La plantilla necesita el motor XeLaTeX\footnote{Para más información sobre XeLaTeX visita \url{https://es.sharelatex.com/learn/XeLaTeX}} (el más recomendable actualmente), por lo que si el programa que utilizas compila la plantilla con el motor pdfLaTeX\footnote{También puedes buscar más información en internet} (el más habitual pero menos potente) debes cambiarlo por XeLaTeX en las opciones del programa. Si no sabes como hacerlo busca en el manual del programa o en google.
%\section{Estilos de texto}
%
%A continuación se muestran ejemplos de distintos estilos de texto:
%
%\begin{itemize}
%	\item \textbackslash textit\{Cursiva\} $\rightarrow$ \textit{Cursiva}
%	\item \textbackslash emph\{Cursiva 2\} $\rightarrow$ \emph{Cursiva 2}
%	\item \textbackslash textbf\{Negrita\} $\rightarrow$ \textbf{Negrita}
%	\item \textbackslash texttt\{Monoespacio\} $\rightarrow$ \texttt{Monoespacio}
%	\item \textbackslash textsc\{Mayúsculas capitales\} $\rightarrow$ \textsc{Mayúsculas capitales}
%	\item \textbackslash uppercase\{Todo mayúsculas\} $\rightarrow$ \uppercase{Todo mayúsculas} 
%\end{itemize}
%
% \section{Acrónimos}
% Ahora vamos a ver cómo se ponen los acrónimos.
% 
% La norma dice que la primera vez que aparece un acrónimo debe ponerse su fórmula completa, es decir lo que significa, al lado del acrónimo. Después de ello, podemos usar sólo el acrónimo salvo cuando consideremos que debemos volver a usar la fórmula completa por alguna razón de legibilidad.
% 
% ¿Cómo llevar la cuenta de cuándo es la primera vez que ponemos el acrónimo? si hacemos cambios en el doc es fácil que perdamos esa información así que lo mejor es que sea el propio \LaTeX~el que lleve esa cuenta. Para ello tenemos que hacer dos cosas:
% \begin{description}
% \item[Primero:] creamos la entrada del acrónimo en el fichero acronimos.tex. Revisa los comentarios de su cabecera para saber cómo crear esa entrada. Básicamente lo que hacemos allí es poner la ``fórmula corta'' y la ``fórmula larga'' del acrónimo es decir, el propio acrónimo y su significado
% \item[Segundo:] escribimos en el texto el acrónimo SIEMPRE diciendo que es un acrónimo y el tipo de fórmula que queremos usar. Por ejemplo, si siempre que queremos hacer referencia al IEEE escribimos \begin{lstlisting}[style=Latex-color]
% \gls{ieee}
% \end{lstlisting}  se consigue que la primera vez que aparezca el acrónimo ponga las fórmulas larga y corta y en las siguientes ocasiones sólo aparecerá la corta.
% \end{description}
% 
% Aquí va un ejemplo:
% 
% Si escribimos:
% 
%\begin{lstlisting}[style=Latex-color]
% El \gls{ieee} es una institución muy importante en el mundo de la
% ingeniería.  El \gls{ieee} lleva marcando normas y protocolos desde
% hace mucho tiempo.  Pero el \gls{ieee} no está solo en esta tarea. 
% Además del \gls{ieee} hay muchas otras instituciones para ello.  \end{lstlisting}
% 
% Obtendremos: 
% 
%El \gls{ieee} es una institución muy importante en el mundo de la
% ingeniería.  El \gls{ieee} lleva marcando normas y protocolos desde
% hace mucho tiempo.  Pero el \gls{ieee} no está solo en esta tarea. 
% Además del \gls{ieee} hay muchas otras instituciones para ello.
% 
% \section{Tareas por hacer}
% 
% En esta plantilla se ha incluido un paquete para incluir notas/comentarios en el texto para recordar partes que hay que revisar o terminar de desarrollar. El uso es sencillo, el manual para conocer todos los comandos se encuentra en \url{http://osl.ugr.es/CTAN/macros/latex/contrib/todonotes/todonotes.pdf}, a continuación se muestran algunos ejemplos:
% \\
%\par Para incluir un comentario sobre el texto:
%
%\begin{lstlisting}[style=Latex-color]
%	Recomiendo utilizar programas LaTeX que permitan trabajar con sistema de archivos para poder editar el conjunto de capítulos en la misma ventana. Este tipo de función lo tienen programas como TexStudio, es multiplataforma. \todo{Incluir más ejemplos de programas}
%\end{lstlisting}
%
%\LaTeX~genera lo siguiente:
%\\
%\par Recomiendo utilizar programas LaTeX que permitan trabajar con sistema de archivos para poder editar el conjunto de capítulos en la misma ventana. Este tipo de función lo tienen programas como TexStudio, es multiplataforma.
%\todo{Incluir más ejemplos de programas}
%\vspace{1em}
%\noindent\hrule
%\vspace{1em}
%\par Para incluir un comentario sobre el texto pero dentro del texto:
%
%\begin{lstlisting}[style=Latex-color]
%	Recomiendo utilizar programas LaTeX que permitan trabajar con sistema de archivos para poder editar el conjunto de capítulos en la misma ventana. Este tipo de función lo tienen programas como TexStudio, es multiplataforma. \todo[inline]{Incluir más ejemplos de programas}
%\end{lstlisting}
%
%\LaTeX~genera lo siguiente:
%\\
%\par Recomiendo utilizar programas LaTeX que permitan trabajar con sistema de archivos para poder editar el conjunto de capítulos en la misma ventana. Este tipo de función lo tienen programas como TexStudio, es multiplataforma. \todo[inline]{Incluir más ejemplos de programas}
%\vspace{1em}
%\noindent\hrule
%\vspace{1em}
%\par También se puede dejar indicado donde falta una imagen o figura, para incluirla más adelante del siguiente modo:
%
%\begin{lstlisting}[style=Latex-color]
%\missingfigure{Añadir gráfica de rendimiento}	
%\end{lstlisting}
%
%\LaTeX~genera lo siguiente:
%\\
%\missingfigure{Añadir gráfica de rendimiento}	



