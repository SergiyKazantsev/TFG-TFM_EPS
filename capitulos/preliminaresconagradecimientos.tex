%%%%%%%%%%%%%%%%%%%%%%%%%%%%%%%%%%%%%%%%%%%%%%%%%%%%%%%%%%%%%%%%%%%%%%%%
% Plantilla TFG/TFM
% Escuela Politécnica Superior de la Universidad de Alicante
% Realizado por: Jose Manuel Requena Plens
% Contacto: info@jmrplens.com / Telegram:@jmrplens
%%%%%%%%%%%%%%%%%%%%%%%%%%%%%%%%%%%%%%%%%%%%%%%%%%%%%%%%%%%%%%%%%%%%%%%%

\chapter*{Preámbulo}
\thispagestyle{empty}
%Poner aquí un texto breve que debe incluir entre otras:
%\begin{quote}
%``las razones que han llevado a la realización del estudio, el tema, la finalidad y el alcance y también los agradecimientos por las ayudas, por ejemplo apoyo económico (becas y subvenciones) y las consultas y discusiones con los tutores y colegas de trabajo. \citep{UNE50136:97}''
%\end{quote}
El desarrollo de este proyecto se llevó a cabo motivado tanto por la investigación de las tecnologías informáticas como por el interés personal sobre la música y harmonía. Actualmente, los niños y adolescentes que cursan estudios musicales asisten a escuelas de música, donde se imparten los conceptos básicos de la estructura de las partituras, el lenguaje musical y el uso de los instrumentos.

\par Este trabajo tiene como objetivo facilitar el aprendizaje de música para los alumnos a través de una aplicación multiplataforma a la cual se podrá acceder desde un dispositivo móvil o un navegador, principalmente durante el tiempo libre o a la hora de hacer los deberes músicales en casa. Para ello, se pretende diseñar una interfaz y experiencia de usuario \textit{user-friendly}. Los alumnos podrán poner en práctica los conceptos teóricos aprendidos a través de la realización de tareas cortas basadas en el Aprendizaje Basado en Juegos. Cabe destacar que no es una aplicación para aprender la teoría musical, sino que ha sido creada pensando en el usuario que ya dispone de conocimentos previos, y sirve para el entrenamiento de sus capacidades auditivas y el refuerzo de las lecciones aprendidas en la escuela de música.

\par Tiene como finalidad abordar varios campos del aprendizaje de música, tales como el reconocimiento de las notas, de los intervalos, modos, tonalidades, acordes, etc. dividido en varios niveles accesibles una vez completado el nivel previo. Como el instrumento musical principal, se usará el piano, generando sus sonidos con MIDI y Soundfont.

\par El desarrollo de la aplicación se lleva a cabo haciendo uso de las Metodologías Ágiles, aplicando el desarrollo incremental basado en iteraciones.
Además, se implementa la integración de gestión de usuarios. Dentro de la aplicación, cada usuario podrá registrarse e iniciar sesión con el fin de poder hacer un seguimiento de su evolución dentro de cada apartado de la aplicación (p. ej. de la sección del reconocimiento de intervalos).
\todo{Revisar el alcance una vez hecho el proyecto}

\cleardoublepage %salta a nueva página impar
\chapter{Agradecimientos}

\thispagestyle{empty}
\vspace{1cm}

Agradecimientos...
\todo{Cambiar}
\cleardoublepage %salta a nueva página impar
% Aquí va la dedicatoria si la hubiese. Si no, comentar la(s) linea(s) siguientes
\chapter*{}
\setlength{\leftmargin}{0.5\textwidth}
\setlength{\parsep}{0cm}
\addtolength{\topsep}{0.5cm}

\begin{flushright}
\small\em{
Dedicatoria...
}
\end{flushright}
\todo{Cambiar}

\cleardoublepage %salta a nueva página impar
% Aquí va la cita célebre si la hubiese. Si no, comentar la(s) linea(s) siguientes
\chapter*{}
\setlength{\leftmargin}{0.5\textwidth}
\setlength{\parsep}{0cm}
\addtolength{\topsep}{0.5cm}
\begin{flushright}
\small\em{
Toda la sabiduría humana\\
está contenida en estas dos palabras:\\ 
Esperar y tener esperanza.
}
\end{flushright}
\begin{flushright}
\small{
El Conde de Montecristo, \\
Alexandre Dumas.
}
\end{flushright}
\cleardoublepage %salta a nueva página impar
