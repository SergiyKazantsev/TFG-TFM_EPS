%%%%%%%%%%%%%%%%%%%%%%%%%%%%%%%%%%%%%%%%%%%%%%%%%%%%%%%%%%%%%%%%%%%%%%%%
% Plantilla TFG/TFM
% Escuela Politécnica Superior de la Universidad de Alicante
% Realizado por: Jose Manuel Requena Plens
% Contacto: info@jmrplens.com / Telegram:@jmrplens
%%%%%%%%%%%%%%%%%%%%%%%%%%%%%%%%%%%%%%%%%%%%%%%%%%%%%%%%%%%%%%%%%%%%%%%%
\chapter{Marco Teórico}
\label{marcoteorico}
A continuación se procederá a revisar el estado del arte actual relativo a las aplicaciones de entrenamiento del oído. Se intentará incluir las aplicaciones más populares, tanto de multiplataforma como de una sola plataforma.
\section{Aplicaciones Web}
Las aplicaciones web forman una gran parte del mercado, ya que se pueden usar prácticamente desde cualquier plataforma, si se dispone de un navegador. En el área del aprendizaje de música, se pueden destacar múltiples sitios web. Dentro de este apartado se han tenido en consideración los proyectos más destacables tanto extranjeros como los de los países hispanohablantes.
\subsection{Teoria.com}
Sitio web con licencia de \textit{Creative Commons} desarrollada por José Rodríguez Alvira, profesor del Conservatorio de Música de Puerto Rico.
\par El sitio web dispone de numerosos tutoriales y ejercicios, de los últimos se pueden destacar:
\begin{itemize}
	\item Práctica de reconocimiento de notas, intervalos.
	\item Reconocimiento de escalas y modos
	\item Reconocimiento de acordes y ritmo
	\item Lectura de las partituras musicales
\end{itemize}
Esta lista no abarca todos los ejercicios del sitio web, dispone de ejercicios más específicos para los modos melódicos, armónicos, también para distintos géneros de música.
\par La vista del sitio es la siguiente:
\begin{figure}[H]
	\centering
	\includegraphics[width=10cm]{archivos/teoria}
	\caption{Página de entrenamiento de los intervalos (\href{https://www.teoria.com/es/ejercicios/ie.php}{teoria.com}), vista desde PC.}
	\label{fig:teoria_PC}
\end{figure}
\par Como se puede observar, cada uno de los ejercicios disponibles en el sitio están diseñados para personas avanzadas en música, que pueden ajustar cada aspecto del ejercicio. Por otro lado, podemos destacar una \gls{ux} que resulta bastante difícil de navegar y ajustar los parámetros.
\par A continuación se muestra la misma página desde un dispositivo móvil:
\begin{figure}[H]
	\centering
	\includegraphics[width=4cm]{archivos/teoria_movil}
		\caption{Página de entrenamiento de los intervalos (\href{https://www.teoria.com/es/ejercicios/ie.php}{teoria.com}), vista desde móvil}
	\label{fig:teoria_movil}
\end{figure}
\par En resumen, \href{teoria.com}{Teoria.com} es un sitio web con una licencia permisiva \textit{Creative Commons} y desarrollada por un autor hispanohablante, está disponible en múltiples idiomas. Sus ejercicios son muy personalizables, por lo que pueden provocar pánico entre principiantes, aunque son ideales para estudiantes avanzados. La aplicación permite iniciar sesión para guardar los resultados, aunque no hay progreso ni gamificación. Los aspectos negativos son la baja \gls{ux} y deficiente adaptabilidad para los dispositivos móviles.
\subsection{Tonedear.com}
Tonedear.com es un sitio web enfocado al entrenamiento del oído, es decir, no dispone de lecciones teóricas ni explicaciones. Tampoco hay un sistema de inicio de sesión o del seguimiento de progreso.
\begin{figure}[htb]
	\centering
	\includegraphics[width=4cm]{archivos/tonedear}
	\caption{Página de entrenamiento de los intervalos (\href{https://tonedear.com/ear-training/intervals}{tonedear.com}), vista desde móvil}
	\label{fig:tonedear}
\end{figure}
\par En esta figura se muestra la vista del sitio desde un dispositivo móvil, se puede destacar una mejor \gls{ux} que en Teoria.com. Los ejercicios de Toned ear son básicos y accesibles para personas incluso sin estudios previos en música.	
\par Por otro lado, en el sitio web existen enlaces a Google Play y App Store para descargar su aplicación nativa para Android o iOS, aunque al parecer no están siendo mantenidas por el autor: la versión de Android ya no está disponible, la versión de iOS tiene una calificación muy baja en App Store por la existencia de bugs y errores, por lo cual no se ha tenido en cuenta en el apartado de aplicaciones multiplataforma.
\subsection{Musictheory.net}
Music Theory es un sitio web de aprendizaje de la música, permite estudiar teoría y entrenar el oído de forma práctica mediante los ejercicios parecidos a otros sitios web, aunque de una forma muy básica y poco personalizable. Music Theory, a diferencia de los otros sitios web, presenta la mejor \gls{ux} en comparación con otros proyectos vistos en esta sección.
\begin{figure}[H]
	\centering
	\includegraphics[width=7cm]{archivos/musictheory}
	\caption{Página de entrenamiento de los intervalos (\href{https://www.musictheory.net/exercises}{musictheory.net})}
	\label{fig:musictheory1}
\end{figure}
\begin{figure}[H]
	\centering
	\includegraphics[width=4cm]{archivos/musictheory3}
	\caption{Página de entrenamiento de los intervalos (\href{https://www.musictheory.net/exercises/ear-interval}{musictheory.net}), vista desde móvil}
	\label{fig:musictheory3}
\end{figure}
Se puede observar que aunque la vista web proporciona una amigable \gls{ux}, no se extiende a los dispositivos móviles, donde es prácticamente imposible ver e interactuar con los botones. Sin embargo, existe una aplicación nativa (Tenuto) desarrollada por el equipo de Music Theory, disponible solo para dispositivos iOS, que resuelve este problema.
%Hacer una lista es simple en \LaTeX. Para ello has de crear un entorno (así se llama) itemize con
%\begin{lstlisting}[style=Latex-color]
%\begin{itemize}
%...
%\end{itemize}
%\end{lstlisting}
%Y dentro de esa estructura, añadir cada elemento de la lista precedido de 
%\begin{lstlisting}[style=Latex-color]
%\item primer ítem de lista
%\item segundo ítem de lista
%...
%\item ultimo ítem de lista
%\end{lstlisting}
%
%Es importante que revises este texto tal como aparece en la plantilla y relaciones el aspecto que tiene el PDF final con cómo está escrito el documento \LaTeX.
%\vspace{1em}
%\noindent\hrule
%\vspace{1em}
%
%Aquí va una lista con subtérminos:
%\begin{lstlisting}[style=Latex-color]
%	\begin{itemize}
%    \item Ingeniería Informática.
%    \item Ingeniería Sonido e Imagen en Telecomunicación.
%    \item Ingeniería Multimedia.
%         \subitem Mención: Creación y ocio digital.
%         \subitem Mención: Gestión de Contenidos.
%	\end{itemize}
%\end{lstlisting}
%
%El resultado es el siguiente:
%\begin{itemize}
%    \item Ingeniería Informática.
%    \item Ingeniería Sonido e Imagen en Telecomunicación.
%    \item Ingeniería Multimedia.
%         \subitem Mención: Creación y ocio digital.
%         \subitem Mención: Gestión de Contenidos.
%\end{itemize}
%\vspace{1em}
%\noindent\hrule
%\vspace{1em}
%Aquí va una lista con subtérminos pero numerada:
%\begin{lstlisting}[style=Latex-color]
%\begin{enumerate}
%    \item Ingeniería Informática.
%    \item Ingeniería Sonido e Imagen en Telecomunicación.
%    \item Ingeniería Multimedia.
%    \begin{enumerate}
%         \item Mención: Creación y ocio digital.
%         \item Mención: Gestión de Contenidos.
%   	\end{enumerate}
%\end{enumerate}
%\end{lstlisting}
%
%El resultado es el siguiente:
%\begin{enumerate}
%    \item Ingeniería Informática.
%    \item Ingeniería Sonido e Imagen en Telecomunicación.
%    \item Ingeniería Multimedia.
%    \begin{enumerate}
%         \item Mención: Creación y ocio digital.
%         \item Mención: Gestión de Contenidos.
%   	\end{enumerate}
%\end{enumerate}
\section{Aplicaciones Móvil y Multiplataforma}
La expansión del mercado de smartphones a partir de los años 2010 llevó a la creación de muchas aplicaciones multiplataforma, entre ellas están las aplicaciones para los músicos y artistas. Se pueden usar para grabar audios, afinar instrumentos y, finalmente, aprender música. En este apartado se muestra el estudio de diversas aplicaciones que incluyen algún aspecto relacionado con el entrenamiento del oído.
\subsection{Perfect Ear}
Perfect Ear es una aplicación disponible para Android y iOS, que se posiciona como una completa escuela de música dentro del teléfono móvil. Esto incluye las lecciones teóricas, el entrenamiento del oído, canto y solfeo.
\par El aspecto visual de la aplicación es el siguiente:
\begin{figure}[H]
	\centering
	\includegraphics[width=3cm]{archivos/perfectear1}
	\caption{Pantalla principal \citep{educkapps}}
	\label{fig:perfectear}
\end{figure}
\begin{figure}[h]
	\centering
	\begin{subfigure}[b]{0.4\textwidth}
		\centering
		\includegraphics[width=3cm]{archivos/perfectear3}
		\caption{Pantalla de los ejercicios del entrenamiento del oído \citep{educkapps}}
		\label{fig:perfectear1}
	\end{subfigure}
	~ % Añadir el espacio deseado, si se deja la linea en blanco la siguiente subfigura ira en una nueva linea
	\begin{subfigure}[b]{0.4\textwidth}
		\centering
		\includegraphics[width=3cm]{archivos/perfectear2}
		\caption{Entrenamiento del reconocimiento de los intervalos \citep{educkapps}}
		\label{fig:perfectear2}
	\end{subfigure}
	\caption{Pantallas del entrenamiento del oído}
\end{figure}
La aplicación tiene una apariencia muy estética, buena \gls{ux}, sistema de inicio de sesión y el seguimiento de progreso. De los aspectos negativos se puede mencionar el acceso a todas las aplicaciones gratuitas desde el momento de registro (no impide empezar con la lección más avanzada de la sección), aunque la mayoría de los entrenamientos está cerrada al usuario detrás de un \textit{paywall}, por lo que perdemos la mayor parte de la aplicación si no pagamos la cuenta premium.
% Puedes realizar una lista de conceptos con su definición del siguiente modo:
% 
%\begin{lstlisting}[style=Latex-color]
%\begin{description} % Inicio de la lista
% 	\item[MAPP XT:] Programa desarrollado por \textit{Meyer Sound} para el diseño y ajuste de sistemas formados por altavoces de su marca.
%  	\begin{description} % Realiza una lista dentro de la lista
%  		\item[Ventajas:]~ 
%  		El programa permite realizar múltiples ajustes tal como se podría realizar en la realidad con un procesador real.
%  	
%  		Permite analizar la fase recibida en cualquier punto y compararla con otras mediciones.
%  	
%  		Dispone de varios tipos de filtros, inversiones de fase, etc.
%  		\item[Inconvenientes:]~ 
%  		No existe una lista global de los altavoces ubicados en el plano, por lo tanto solo se pueden editar seleccionándolos sobre el plano.
%  	
%  		Sólo permite diseñar en 2 dimensiones, principalmente sobre la vista lateral ya que los array de altavoces no permite voltearlos.
%  	\end{description}
%\end{description}
%\end{lstlisting}
%
% Y \LaTeX~genera lo siguiente:
% 
%\begin{description} % Inicio de la lista
% 	\item[MAPP XT:] Programa desarrollado por \textit{Meyer Sound} para el diseño y ajuste de sistemas formados por altavoces de su marca.
%  	\begin{description} % Realiza una lista dentro de la lista
%  		\item[Ventajas:]~ 
%  		El programa permite realizar múltiples ajustes tal como se podría realizar en la realidad con un procesador real.
%  	
%  		Permite analizar la fase recibida en cualquier punto y compararla con otras mediciones.
%  	
%  		Dispone de varios tipos de filtros, inversiones de fase, etc.
%  		\item[Inconvenientes:]~ 
%  		No existe una lista global de los altavoces ubicados en el plano, por lo tanto solo se pueden editar seleccionándolos sobre el plano.
%  	
%  		Sólo permite diseñar en 2 dimensiones, principalmente sobre la vista lateral ya que los array de altavoces no permite voltearlos.
%  	\end{description}
%\end{description}

\subsection{Functional Ear Trainer}
Functional Ear Training es una aplicación que utiliza métodos heterodoxos de entrenamiento del oído, que no siguen la estructura vista en el capítulo anterior \citep{rd-1577-2006}. A diferencia del resto de aplicaciones, Functional Ear Trainer enseña a reconocer tonos dentro de canciones y melodías al escucharlas, incluso sin tener estudios previos en música. El programa utilizado es llamado el Método de Alain Benbassat. Este método se enfoca a reconocer tonos relativos con el fin de transcribir melodías y tocar música usando el oído interno \citep{getmusictools}.
\par La funcionalidad más interesante de la aplicación es el "dictado musical", en el cual el estudiante es obligado a reconocer todas las notas de una secuencia (melodía) y transcribir cada una en una hoja de partituras, lo que entrena a la vez las habilidades auditivas y el solfeo \citep{garcia-gil-2022}.
\begin{figure}[H]
	\centering
	\includegraphics[width=4cm]{archivos/fet}
	\caption{El dictado musical en la aplicación Functional Ear Trainer}
	\label{fig:fet}
\end{figure}
La aplicación esta diseñada sobre todo para los principiantes, y en la figura \label{fet} proporcionada se puede observar un buen diseño de la aplicación, como la gamificación y seguimiento del progreso de la misma. A medida que el estudiante avanza, se le habilita más opciones de notas, por lo cual, mayor complejidad.
\newpage
\subsection{Duolingo}
Duolingo es una aplicación multiplataforma de aprendizaje, que incorpora aspectos innovadores de enseñanza, tales como el Aprendizaje Basado en Juegos, o en otras palabras, la gamificación del proceso de aprendizaje. Otro aspecto interesante de esta aplicación es el seguimiento del progreso y unas "rachas" que determinan el número de días seguidos de acceso a la aplicación. Todo esto hizo que Duolingo se ha convertido en la plataforma definitiva de aprendizaje de idiomas, y es usada por 103 millones de personas mensualmente \citep{duolingo-2026}.
\par Aunque Duolingo es una plataforma generalmente usada para aprender idiomas, en los últimos años ha incorporado nuevos cursos de enseñanza de ajedrez y música. En esta sección se contemplarán los aspectos positivos y negativos de Duolingo en comparación con el programa de enseñanza de música establecido por el \cite{rd-1577-2006}.
\begin{figure}[H]
	\centering
	\begin{subfigure}[b]{0.4\textwidth}
		\centering
		\includegraphics[width=3cm]{archivos/duolingo1}
		\caption{Pantalla principal del curso de música (Duolingo) \citep{duolingo-2026-2}.}
		\label{fig:duolingo1}
	\end{subfigure}
	~ % Añadir el espacio deseado, si se deja la linea en blanco la siguiente subfigura ira en una nueva linea
	\begin{subfigure}[b]{0.4\textwidth}
		\centering
		\includegraphics[width=7cm]{archivos/duolingo2}
		\caption{Pantalla de una lección del curso de música (Duolingo) \citep{duolingo-2026-2}.}
		\label{fig:duolingo2}
	\end{subfigure}
	\caption{Pantallas del curso de música de Duolingo.}
\end{figure}
\par Se puede observar que Duolingo tiene el mejor diseño e interactivo, fruto de mayores ingresos y usuarios en comparación con las otras aplicaciones vistas hasta ahora. Por otro lado, el programa del entrenamiento del oído es inferior a las alternativas, ya que el curso que ofrece Duolingo está diseñado para enseñar cómo tocar el piano y como leer y redactar partituras musicales, pero no incluye nada de reconocimiento de los intervalos, acordes, modos, etc. Por lo cual, no es compatible con el programa de enseñanza tradicional y no es relevante para reforzar las lecciones teóricas del conservatorio.
\newpage
\section{Resumen de las aplicaciones de competencia}
Todas las aplicaciones vistas en este capítulo incluyen tanto los aspectos positivos como algunos negativos, con los cuales se puede sacar conclusiones de los puntos de mejora para el proyecto a desarrollar.
\begin{table}[H]
	\centering
	\begin{tabular}{|C{4cm}|C{2cm}|C{2cm}|C{3cm}|C{3cm}|}
		\hline
		\multicolumn{5}{|c|}{\textbf{Comparativa: Aplicaciones de competencia}} \\ \hline
		\textbf{Aplicación} & \textbf{Nivel de enseñanza} & \textbf{Personalización} & \textbf{\gls{ux}} & \textbf{Seguimiento del progreso} \\ \hline
		
		\textbf{Teoria.com} & \cellcolor{green!40}Muy alto & \cellcolor{green!40}Muy alta & \cellcolor{red!30}Muy baja & \cellcolor{red!30}No \\ \hline
		
		\textbf{Tonedear.com} & \cellcolor{green!20}Alto & \cellcolor{yellow!30}Media & \cellcolor{yellow!30}Media & \cellcolor{red!30}No \\ \hline
		
		\textbf{Musictheory.net} & \cellcolor{yellow!30}Medio & \cellcolor{orange!40}Baja & \cellcolor{red!30}Muy baja & \cellcolor{red!30}No \\ \hline
		
		\textbf{Perfect Ear} & \cellcolor{green!20}Alto & \cellcolor{orange!40}Baja & \cellcolor{green!40}Alta & \cellcolor{green!40}Sí \\ \hline
		
		\textbf{Functional Ear Trainer} & \cellcolor{yellow!30}Medio & \cellcolor{yellow!30}Media & \cellcolor{yellow!30}Media & \cellcolor{green!40}Sí \\ \hline
		
		\textbf{Duolingo} & \cellcolor{red!30}Muy bajo & \cellcolor{orange!40}Baja & \cellcolor{green!40}Muy Alta & \cellcolor{green!40}Sí \\ \hline
	\end{tabular}
	\caption{Comparativa entre aplicaciones de la competencia}
	\label{tabla-apps}
\end{table}
Es aparente que las aplicaciones vistas, en general, están polarizadas en dos campos: o son diseñadas para profesionales con una alta personalización y nivel de enseñanza, o tienen una alta \gls{ux}, pero no entrenan el oído acorde con el programa de los conservatorios. También, las aplicaciones disponibles en la Web no incluyen la funcionalidad del inicio de sesión y el seguimiento del progreso, lo cual puede disminuir el tiempo de uso por parte del estudiante.
\section{Propuesta de valor única}
El objetivo de este trabajo es combinar los aspectos positivos de ambos extremos vistas en la tabla \ref{tabla-apps}. Esto es, el proyecto a desarrollar tendrá que ofrecer el reforzamiento necesario para los estudiantes de los conservatorios y estar alineado con el plan de estudios expuesto en el \cite{rd-1577-2006} y con los conceptos explicados en el capítulo de Introducción. El estudiante, una vez determinado su nivel de estudios, podrá acceder a las secciones progresivamente, aplicando los principios del Aprendizaje Basado en Juegos, obteniendo una puntuación por cada lección realizada, y con la posibilidad de poder hacer un seguimiento de su progreso dentro de la aplicación, todo de forma completamente gratuita.
\par La aplicación, realizada únicamente con fines académicos, estará enfocada a los estudiantes procedentes de familias de bajos ingresos y a los entusiastas que quieren reforzar las lecciones teóricas aprendidas en el conservatorio. En consecuencia, podrán entrenar el oído y reforzar sus capacidades incluso sin tener acceso a los instrumentos o los servicios de los profesores particulares.
\par Por otro lado, la aplicación tendrá que ser accesible tanto a través de la Web, como a través del sistema operativo Android. Teniendo en cuenta el público objetivo y la ausencia de los dispositivos Apple por parte del autor, la compatibilidad con el sistema operativo iOS no será prioritaria dentro del proceso de desarrollo.
\section{Lean Canvas}
Lean Canvas es una plantilla de desarrollo que es frecuentemente utilizada por los start-ups, y que permite desglosar el proyecto en partes concretas, tales como:
\begin{itemize}
	\item \textbf{El problema:} Qué se pretende resolver.
	\item \textbf{La solución:} Una solución creativa al problema.
	\item \textbf{Métricas clave:} ¿Qué significa tener éxito?
	\item \textbf{Propuesta de valor única:} Valor que obtendrá el cliente al elegir el producto.
	\item \textbf{Canales de distribución:} Forma de llegar a los clientes.
	\item \textbf{El público objetivo:} Definir los clientes potenciales.
	\item \textbf{Los costos}
	\item \textbf{El flujo de ingresos:}
	\item \textbf{Ventaja especial:} La ventaja frente a los competidores.
\end{itemize}
\par A continuación se proporciona el Lean Canvas del proyecto a desarrollar:
\begin{figure}[H]
	\centering
	\includegraphics[width=15cm]{archivos/lean_canvas}
	\caption{Lean Canvas de la aplicación a desarrollar}
	\label{fig:lean_canvas}
\end{figure}